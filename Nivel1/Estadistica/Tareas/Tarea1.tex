\documentclass[12pt]{article}


% -------------------- PAQUETES --------------------
\usepackage[utf8]{inputenc}
\usepackage[spanish]{babel}
\usepackage[margin=2.54cm]{geometry}
\usepackage{graphicx}
\usepackage{xcolor}


% -------------------- CARGA DE ARCHIVOS EXTERNOS --------------------
\input{../../../RecursosGlobales/Docs/FormatoDocs.tex}
\input{../../../RecursosGlobales/Docs/PortadaTareasDef}


% -------------------- DEFINICIÓN DE LA PORTADA --------------------
\rutaLogo{../../../RecursosGlobales/Img/logo_tec_azuay.png}
\tema{\\ \vspace{1cm} Ejercicios propuestos sobre muestreo \\ \vspace{1.7cm}}
\etiquetaAutores{Alumno:}
\alumno{Eduardo Mendieta \vspace{1cm}}
\materia{Fundamentos de Estadística \vspace{1cm}}
\docente{Ing. Héctor Mejía, Mgtr. \vspace{1cm}}
\ciclo{Primer Ciclo - M1A \vspace{1.1cm}}
\fecha{18 de junio de 2024 \vspace{1cm}}
\periodo{Abril 2024 - Agosto 2024}



% -------------------- INFORME --------------------
\begin{document}

    \input{../../../RecursosGlobales/Docs/PortadaTareas}
  
    \section*{\centering Ejercicios propuestos}

        \vspace{0.6cm}
        \noindent \textbf{Resolver los ejercicios a mano y subir en un archivo  Pdf:}

        \vspace{0.6cm}
        \begin{itemize}
            \item \textbf{Ejercicio 1:} Un investigador desea estimar la proporción de estudiantes universitarios que utilizan el transporte público para ir a la universidad. Se desea una confianza del 95\% y un margen de error del 5\%. ¿Cuál es el tamaño de muestra necesario si se asume que la proporción esperada es del 50\%?
            
            
            \item \textbf{Ejercicio 2:} La empresa COLINEAL tiene 800 empleados y quiere realizar una encuesta sobre la satisfacción laboral. Se desea un nivel de confianza del 95\% y un margen de error del 5\%. ¿Cuál es el tamaño de la muestra necesario?
            
            
            \item \textbf{Ejercicio 3:} Una organización quiere estimar el porcentaje de hogares que tienen acceso a internet con una confianza del 99\% y un margen de error del 3\%. Si se asume que la proporción esperada es del 70\%, ¿cuál es el tamaño de la muestra necesario?
            
            
            \item \textbf{Ejercicio 4:} El ISTA tiene 1500 estudiantes y quiere hacer una encuesta sobre la satisfacción con las instalaciones deportivas. Se desea un nivel de confianza del 90\% y un margen de error del 4\%. ¿Cuál es el tamaño de la muestra necesario?
            
            
            \item \textbf{Ejercicio 5:} Un investigador está interesado en conocer la proporción de adultos que hacen ejercicio regularmente. Desea un nivel de confianza del 95\% y un margen de error del 2\%. ¿Cuál es el tamaño de la muestra necesario si se asume que la proporción esperada es del 40\%?
            
            
            \item \textbf{Ejercicio 6:} Una universidad tiene 3000 alumnos y desea realizar una encuesta para saber cuántos estudiantes están interesados en nuevos cursos de tecnología. Se desea un nivel de confianza del 95\% y un margen de error del 3\%. ¿Cuál es el tamaño de la muestra necesario?
            

        \end{itemize}

\end{document}