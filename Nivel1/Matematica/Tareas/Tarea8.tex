\documentclass[12pt]{article}


% -------------------- PAQUETES --------------------
\usepackage[utf8]{inputenc}
\usepackage[spanish]{babel}
\usepackage[margin=2.54cm]{geometry}
\usepackage{graphicx}
\usepackage{xcolor}


% -------------------- CARGA DE ARCHIVOS EXTERNOS --------------------
% ----------------- UTILIDADES PARA DAR UN MEJOR FORMATO DE DOCUMENTO -----------------  


\definecolor{azul}{rgb}{0.0039, 0.3098, 0.6196}


% Formato para el indice general ...........
\makeatletter
    \renewcommand{\@dotsep}{1.5}
    \renewcommand{\l@section}{\@dottedtocline{1}{1.5em}{2.3em}}
    \renewcommand{\l@subsection}{\@dottedtocline{2}{3.8em}{3.2em}}
    \renewcommand{\l@subsubsection}{\@dottedtocline{3}{7.0em}{4.1em}}
\makeatother

% --------- COMANDOS PERSONALIZADOS PARA LA PORTADA DE LAS TAREAS, TRABAJOS Y PROYECTOS ---------

\newcommand{\rutaLogo}[1]{\newcommand{\RutaLogo}{#1}}
\newcommand{\tema}[1]{\newcommand{\Tema}{#1}}
\newcommand{\etiquetaAutores}[1]{\newcommand{\EtiquetaAutores}{#1}}
\newcommand{\alumno}[1]{\newcommand{\Alumno}{#1}}
\newcommand{\materia}[1]{\newcommand{\Materia}{#1}}
\newcommand{\docente}[1]{\newcommand{\Docente}{#1}}
\newcommand{\ciclo}[1]{\newcommand{\Ciclo}{#1}}
\newcommand{\fecha}[1]{\newcommand{\Fecha}{#1}}
\newcommand{\periodo}[1]{\newcommand{\Periodo}{#1}}



% -------------------- DEFINICIÓN DE LA PORTADA --------------------
\rutaLogo{../../../RecursosGlobales/Img/logo_tec_azuay.png}
\tema{\\ \vspace{1cm} Guía Práctica N°1 - Introducción a la lógica proposicional: conceptos y estructuras básicas  \\ \vspace{1.7cm}}
\etiquetaAutores{Alumno:}
\alumno{Eduardo Mendieta \vspace{1cm}}
\materia{Matemática \vspace{1cm}}
\docente{Lcda. Vilma Duchi, Mgtr. \vspace{1cm}}
\ciclo{Primer Ciclo - M1A \vspace{1.1cm}}
\fecha{16 de junio de 2024 \vspace{1cm}}
\periodo{Abril 2024 - Agosto 2024}



% -------------------- INFORME --------------------
\begin{document}

    \begin{titlepage}

    \centering

    \includegraphics[width=0.11\textwidth]{\RutaLogo} 

    \vspace{0.3cm}
    \textcolor{azul}{\Large \textbf{Instituto Superior Universitario Tecnológico del Azuay \\}}
    \vspace{0.3cm}
    \textcolor{azul}{\Large \textbf{Tecnología Superior en Big Data}}
    
    % 1. ---------------- TEMA -------------------------
    
    {\Large\textbf{\Tema}}
    
    % 2. ---------------- AUTOR(ES) -------------------------
    \textcolor{azul}{\large \textbf{\EtiquetaAutores} \\}
    \vspace{0.3cm}
    {\large \Alumno}

    % 3. ---------------- MATERIA -------------------------
    \textcolor{azul}{\large \textbf{Materia:} \\}
    \vspace{0.3cm}
    {\large \Materia}


    % 3. ---------------- DOCENTE -------------------------
    \textcolor{azul}{\large \textbf{Docente:} \\}
    \vspace{0.3cm}
    {\large \Docente}


    % 3. ---------------- Ciclo -------------------------
    \textcolor{azul}{\large \textbf{Ciclo:} \\}
    \vspace{0.3cm}
    {\large \Ciclo}


    % 3. ---------------- FECHA -------------------------
    \textcolor{azul}{\large \textbf{Fecha:} \\}
    \vspace{0.3cm}
    {\large \Fecha}

    % 3. ---------------- PERIODO -------------------------
    \textcolor{azul}{\large \textbf{Periodo Académico:} \\}
    \vspace{0.3cm}
    {\large \Periodo}
 
\end{titlepage}

  
    \section*{\centering Guía práctica N°1 - Unidad 2\\Aplicaciones Prácticas}

        \vspace{0.6cm}
        \begin{itemize}
            % PARTE 1: -----------------------------------------------------
            \item \textbf{Problemas de la Vida Real:} \\Aplicar la lógica proposicional para modelar y resolver 3 problemas de la vida cotidiana. Una vez modelada las situaciones elabore la tabla de verdad y en otro apartado simplifique las aplicando las leyes de lógica proposicional.
            
            
            % PARTE 2: -----------------------------------------------------
            \item \textbf{ Poner en práctica lo aprendido: Modelar las situaciones en expresiones lógicas.}\\ Dadas las siguientes situaciones formalice a lenguaje proposicional y elabore las tablas de verdad: 
            
            \begin{enumerate}
                % Ejercicio N°1: ..........................................
                \item Para que se organice un evento exitoso en un parque, se deben cumplir varias condiciones: debe ser un día soleado, los permisos del gobierno deben estar aprobados, el equipo de sonido debe estar disponible y el catering debe estar confirmado. Además, si llueve, el evento se trasladará a un auditorio, pero solo si el auditorio está disponible.
                
                % Ejercicio N°2: ..........................................
                \item Para que un proyecto de investigación sea aceptado, debe cumplir con ciertos criterios: el proyecto debe ser innovador, debe contar con la aprobación del comité de ética, y debe tener financiamiento asegurado. Además, si el proyecto involucra experimentación con humanos, debe tener la autorización de los sujetos participantes y el respaldo de un hospital.
                
                % Ejercicio N°3: ..........................................
                \item Para ser admitido en un programa de posgrado, un estudiante debe tener una licenciatura, haber pasado un examen de admisión, y contar con una carta de recomendación de un profesor. Además, si el estudiante no tiene una licenciatura en el campo específico del programa, debe haber completado cursos de nivelación.
                
                % Ejercicio N°4: ..........................................
                \item Para implementar una plataforma de análisis de Big Data, se deben cumplir varias condiciones: el almacenamiento debe estar configurado, los datos deben estar limpios y preparados, el equipo de análisis debe estar capacitado, y las herramientas de visualización deben estar integradas. Además, si los datos incluyen información sensible, se deben cumplir las normativas de privacidad y seguridad.
                
                % Ejercicio N°5: ..........................................
                \item Una empresa desea predecir la demanda de sus productos para optimizar su cadena de suministro. Para ello, deben cumplirse las siguientes condiciones:
                    \begin{itemize}
                        \item Los datos históricos de ventas deben estar disponibles.
                        \item Si la demanda de un producto ha aumentado en los últimos 3 meses, se considera una tendencia al alza.
                        \item Si la demanda de un producto ha disminuido en los últimos 3 meses, se considera una tendencia a la baja.
                        \item Se desea identificar los productos que tienen una tendencia al alza y una tendencia a la baja para ajustar la producción.
                    \end{itemize}
            \end{enumerate}
        \end{itemize}
\end{document}