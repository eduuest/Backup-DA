\documentclass[12pt]{article}


% -------------------- PAQUETES --------------------
\usepackage[utf8]{inputenc}
\usepackage[spanish]{babel}
\usepackage[margin=2.54cm]{geometry}
\usepackage{graphicx}
\usepackage{xcolor}
\usepackage{venndiagram}


% -------------------- CARGA DE ARCHIVOS EXTERNOS --------------------
% ----------------- UTILIDADES PARA DAR UN MEJOR FORMATO DE DOCUMENTO -----------------  


\definecolor{azul}{rgb}{0.0039, 0.3098, 0.6196}


% Formato para el indice general ...........
\makeatletter
    \renewcommand{\@dotsep}{1.5}
    \renewcommand{\l@section}{\@dottedtocline{1}{1.5em}{2.3em}}
    \renewcommand{\l@subsection}{\@dottedtocline{2}{3.8em}{3.2em}}
    \renewcommand{\l@subsubsection}{\@dottedtocline{3}{7.0em}{4.1em}}
\makeatother

% --------- COMANDOS PERSONALIZADOS PARA LA PORTADA DE LAS TAREAS, TRABAJOS Y PROYECTOS ---------

\newcommand{\rutaLogo}[1]{\newcommand{\RutaLogo}{#1}}
\newcommand{\tema}[1]{\newcommand{\Tema}{#1}}
\newcommand{\etiquetaAutores}[1]{\newcommand{\EtiquetaAutores}{#1}}
\newcommand{\alumno}[1]{\newcommand{\Alumno}{#1}}
\newcommand{\materia}[1]{\newcommand{\Materia}{#1}}
\newcommand{\docente}[1]{\newcommand{\Docente}{#1}}
\newcommand{\ciclo}[1]{\newcommand{\Ciclo}{#1}}
\newcommand{\fecha}[1]{\newcommand{\Fecha}{#1}}
\newcommand{\periodo}[1]{\newcommand{\Periodo}{#1}}



% -------------------- DEFINICIÓN DE LA PORTADA --------------------
\rutaLogo{../../../RecursosGlobales/Img/logo_tec_azuay.png}
\tema{\\ \vspace{1cm} Actividad N°1: Taller de conjuntos - Problemas \\ \vspace{1.7cm}}
\etiquetaAutores{Alumno:}
\alumno{Eduardo Mendieta \vspace{1cm}}
\materia{Matemática \vspace{1cm}}
\docente{Lcda. Vilma Duchi \vspace{1cm}}
\ciclo{Primer Ciclo \vspace{1.1cm}}
\fecha{30 de mayo de 2024 \vspace{1cm}}
\periodo{Abril 2024 - Agosto 2024}



% -------------------- INFORME --------------------
\begin{document}

    \begin{titlepage}

    \centering

    \includegraphics[width=0.11\textwidth]{\RutaLogo} 

    \vspace{0.3cm}
    \textcolor{azul}{\Large \textbf{Instituto Superior Universitario Tecnológico del Azuay \\}}
    \vspace{0.3cm}
    \textcolor{azul}{\Large \textbf{Tecnología Superior en Big Data}}
    
    % 1. ---------------- TEMA -------------------------
    
    {\Large\textbf{\Tema}}
    
    % 2. ---------------- AUTOR(ES) -------------------------
    \textcolor{azul}{\large \textbf{\EtiquetaAutores} \\}
    \vspace{0.3cm}
    {\large \Alumno}

    % 3. ---------------- MATERIA -------------------------
    \textcolor{azul}{\large \textbf{Materia:} \\}
    \vspace{0.3cm}
    {\large \Materia}


    % 3. ---------------- DOCENTE -------------------------
    \textcolor{azul}{\large \textbf{Docente:} \\}
    \vspace{0.3cm}
    {\large \Docente}


    % 3. ---------------- Ciclo -------------------------
    \textcolor{azul}{\large \textbf{Ciclo:} \\}
    \vspace{0.3cm}
    {\large \Ciclo}


    % 3. ---------------- FECHA -------------------------
    \textcolor{azul}{\large \textbf{Fecha:} \\}
    \vspace{0.3cm}
    {\large \Fecha}

    % 3. ---------------- PERIODO -------------------------
    \textcolor{azul}{\large \textbf{Periodo Académico:} \\}
    \vspace{0.3cm}
    {\large \Periodo}
 
\end{titlepage}

  
    \section*{\centering Actividad en clase N°1}

        \begin{enumerate}
            \item \textbf{En un club deportivo, el 80\% de los socios juegan al fútbol y el 40\% al baloncesto. Sabiendo que el 30\% de los socios practican los 2 deportes, calcula la probabilidad de que un socio elegido al azar:\\a) Juegue sólo al fútbol.\\b) Juegue sólo al baloncesto.\\c) Juegue al fútbol y al baloncesto.\\d) No juegue a ninguno de los dos deportes.}
                
                \vspace{1cm}
                \begin{venndiagram2sets}[labelA = F, labelB = B, labelAB = \textbf{30\%}, tikzoptions = {scale = 1.5}]
                    \fillACapB
                \end{venndiagram2sets}

                \begin{enumerate}
                    \item $U = 100\%$
                    \item $F \cap B = 30\%$
                    \item $F - B = 80\% - (F \cap B) = 80\% - 30\% = 50\%$
                    \item $B - F = 40\% - (F \cap B) = 40\% - 30\% = 10\%$
                    \item $F \cup B = 50\% + 30\% + 10\% = 90\%$
                    \item $U - (F \cup B) = 100\% - 90\% = 10\%$
                \end{enumerate}

                \vspace{1cm}

                \begin{venndiagram2sets}[labelNotAB = 10\%, labelA = F, labelB = B, labelAB = 30\%, labelOnlyA = 50\%, labelOnlyB = 10\%, tikzoptions = {scale = 1.5}]
                    
                \end{venndiagram2sets}

                \newpage
                \textbf{Respuesta:}

                \begin{enumerate}
                    \item El 50\% juegan sólo fútbol.
                    \item El 10\% juegan sólo baloncesto.
                    \item El 30\% juegan al fútbol y al baloncesto.
                    \item El 10\% no juegan ningún deporte.
                \end{enumerate}

                

            \item \textbf{En un grupo de 30 estudiantes pertenecientes a un curso, 15 no estudiaron Matemáticas y 19 no estudiaron Lenguaje. Si tenemos un total de 12 alumnos que no estudiaron Lenguaje ni Matemáticas. ¿Cuántos alumnos estudian exactamente una de las materias mencionadas?}
            
                \vspace{1cm}
                \begin{venndiagram2sets}[labelNotAB = \textbf{12}, labelA = M, labelB = L, tikzoptions = {scale = 1.5}]
                    \fillNotAorB
                \end{venndiagram2sets}

                \begin{enumerate}
                    \item $U = 30$
                    \item $x = U - (M \cup L) = 12$
                    \item $x + (L - M) = 15\\(L - M) = 15 - x\\(L - M) = 15 - 12 = 3$
                    \item $x + (M - L) = 19\\(M - L) = 19 - x\\(M - L) = 19 - 12 = 7$
                    \item $M \cap L = 30 - 12 - 3 - 7 = 8$
                    \item $A \mathbin{\triangle} B = (L - M) \cup (M - L) = 3 + 7 = 10$ 
                \end{enumerate}

                \begin{venndiagram2sets}[labelNotAB = 12, labelA = M, labelB = L, labelAB = 8, labelOnlyA = 7, labelOnlyB = 3, tikzoptions = {scale = 1.5}]
                 
                \vspace{1cm}
                \end{venndiagram2sets}

                \textbf{Respuesta:} 10 alumnos estudian exactamente una de las materias mencionadas.

               

            \item \textbf{En una investigación hecha a un grupo de 100 estudiantes, la cantidad de personas que estudian idiomas fueron las siguientes: español, 28; alemán, 30; y francés, 42; español y alemán, 8; español y francés, 10; alemán y francés, 5; los 3 idiomas, 3.\\a) ¿Cuántos no estudian nungún idioma?\\b) ¿Cuántos estudiantes tenían el francés como único idioma de estudio?}
            
                \vspace{1cm}
                \begin{venndiagram3sets}[labelA = E, labelB = A, labelC = F, labelABC = 3, tikzoptions = {scale = 1.5}]
                    \fillACapBCapC
                \end{venndiagram3sets}

                \begin{enumerate}
                    \item $X = (E \cap A) \cap F = 3$
                    \item $E \cap A = 8\\(E \cap A) - X = 8 - 3 = 5$
                    \item $E \cap F = 10\\(E \cap F) - X = 10 - 3 = 7$
                    \item $A \cap F = 5\\(A \cap F) - X = 5 - 3 = 2$
                \end{enumerate}

                \vspace{1cm}
                \begin{venndiagram3sets}[labelA = E, labelB = A, labelC = F, labelABC = 3, labelOnlyAB = 5, labelOnlyAC = 7, labelOnlyBC = 2, tikzoptions = {scale = 1.5}]
                    
                \end{venndiagram3sets}

                \begin{enumerate}
                    \item $E = 28\\(E - A) - F = 28 - 5 - 3 - 7 = 13$
                    \item $A = 30\\(A - E) - F = 30 - 5 - 3 -2 = 20$
                    \item $F = 42\\(F - E) - A = 42 - 7 - 3 - 2 = 30$ 
                    \item $ (E \cup A) \cup F = 13 + 20 + 30 + 3 + 5 + 7 + 2 = 80$
                    \item $U = 100\\U - ((E \cup A) \cup F) = 100 - 80 = 20$
                \end{enumerate}

                \vspace{1cm}
                \begin{venndiagram3sets}[labelNotABC = 20, labelA = E, labelB = A, labelC = F, labelABC = 3, labelOnlyAB = 5, labelOnlyAC = 7, labelOnlyBC = 2, labelOnlyA = 13, labelOnlyB = 20, labelOnlyC = 30, tikzoptions = {scale = 1.5}]
                    
                \end{venndiagram3sets}

                \newpage
                \textbf{Respuesta:}

                \begin{enumerate}
                    \item 20 alumnos no estudian ningún idioma.
                    \item 30 estudiantes tenían el francés como único idioma de estudio.
                \end{enumerate}


            \item \textbf{En una reunión se determina que 40 personas son aficionadas al juego, 39 son aficionadas al vino y 48 a las fiestas, además hay 10 personas que son aficionadas al vino, juego y fiestas, existen 9 personas aficionadas al juego y vino solamente, hay 11 personas que son aficionadas al juego solamente y por último 9 a las fiestas y al vino solamente.\\Determinar:\\a) El número de personas que es aficionada al vino solamente.\\b) El número de personas que es aficionada a las fiestas solamente.}
        \end{enumerate}

\end{document}