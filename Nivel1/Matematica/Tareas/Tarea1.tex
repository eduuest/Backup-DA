\documentclass[12pt]{article}


% -------------------- PAQUETES --------------------
\usepackage[utf8]{inputenc}
\usepackage[spanish]{babel}
\usepackage[margin=2.54cm]{geometry}
\usepackage{graphicx}
\usepackage{xcolor}
\usepackage{venndiagram}


% -------------------- CARGA DE ARCHIVOS EXTERNOS --------------------
\input{../../../RecursosGlobales/Docs/FormatoDocs.tex}
\input{../../../RecursosGlobales/Docs/PortadaTareasDef}


% -------------------- DEFINICIÓN DE LA PORTADA --------------------
\rutaLogo{../../../RecursosGlobales/Img/logo_tec_azuay.png}
\tema{\\ \vspace{1cm} Actividad N°1: Taller de conjuntos - Problemas \\ \vspace{1.7cm}}
\etiquetaAutores{Alumno:}
\alumno{Eduardo Mendieta \vspace{1cm}}
\materia{Matemática \vspace{1cm}}
\docente{Lcda. Vilma Duchi \vspace{1cm}}
\ciclo{Primer Ciclo \vspace{1.1cm}}
\fecha{30 de mayo de 2024 \vspace{1cm}}
\periodo{Abril 2024 - Agosto 2024}



% -------------------- INFORME --------------------
\begin{document}

    \input{../../../RecursosGlobales/Docs/PortadaTareas}
  
    \section*{\centering Actividad en clase N°1}

        \begin{enumerate}
            \item \textbf{En un club deportivo, el 80\% de los socios juegan al fútbol y el 40\% al baloncesto. Sabiendo que el 30\% de los socios practican los 2 deportes, calcula la probabilidad de que un socio elegido al azar:\\a) Juegue sólo al fútbol.\\b) Juegue sólo al baloncesto.\\c) Juegue al fútbol y al baloncesto.\\d) No juegue a ninguno de los dos deportes.}
                
                \vspace{1cm}
                \begin{venndiagram2sets}[labelA = F, labelB = B, labelAB = \textbf{30\%}, tikzoptions = {scale = 1.5}]
                    \fillACapB
                \end{venndiagram2sets}

                \begin{enumerate}
                    \item $F \cap B = 30\%$
                    \item $F - (F \cap B) = 80\% - 30\% = 50\%$
                    \item $B - (F \cap B) = 40\% - 30\% = 10\%$
                    \item $F \cup B = 50\% + 30\% + 10\% = 90\%$
                    \item $U - (F \cup B) = 100\% - 90\%$
                \end{enumerate}

                \textbf{Respuesta:}

                \begin{enumerate}
                    \item El 50\% juegan sólo fútbol.
                    \item El 10\% juegan sólo baloncesto.
                    \item El 30\% juegan al fútbol y al baloncesto.
                    \item El 10\% no juegan ningún deporte.
                \end{enumerate}

                \vspace{1cm}

                \begin{venndiagram2sets}[labelNotAB = 10\%, labelA = F, labelB = B, labelAB = 30\%, labelOnlyA = 50\%, labelOnlyB = 10\%, tikzoptions = {scale = 1.5}]
                    
                \end{venndiagram2sets}

            \item \textbf{En un grupo de 30 estudiantes pertenecientes a un curso, 15 no estudiaron Matemáticas y 19 no estudiaron Lenguaje. Si tenemos un total de 12 alumnos que no estudiaron Lenguaje ni Matemáticas. ¿Cuántos alumnos estudian exactamente una de las materias mencionadas?}
            \item \textbf{En una investigación hecha a un grupo de 100 estudiantes, la cantidad de personas que estudian idiomas fueron las siguientes: español, 28; alemán, 30; y francés, 42; español y alemán, 8; español y francés, 10; alemán y francés, 5; los 3 idiomas, 3.\\a) ¿Cuántos no estudian nungún idioma?\\b) ¿Cuántos estudiantes tenían el francés como único idioma de estudio?}
            \item \textbf{En una reunión se determina que 40 personas son aficionadas al juego, 39 son aficionadas al vino y 48 a las fiestas, además hay 10 personas que son aficionadas al vino, juego y fiestas, existen 9 personas aficionadas al juego y vino solamente, hay 11 personas que son aficionadas al juego solamente y por último 9 a las fiestas y al vino solamente.\\Determinar:\\a) El número de personas que es aficionada al vino solamente.\\b) El número de personas que es aficionada a las fiestas solamente.}
        \end{enumerate}

\end{document}