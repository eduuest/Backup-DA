\documentclass[12pt]{article}


% -------------------- PAQUETES --------------------
\usepackage[utf8]{inputenc}
\usepackage[spanish]{babel}
\usepackage[margin=2.54cm]{geometry}
\usepackage{graphicx}
\usepackage{xcolor}
\usepackage{venndiagram}


% -------------------- CARGA DE ARCHIVOS EXTERNOS --------------------
% ----------------- UTILIDADES PARA DAR UN MEJOR FORMATO DE DOCUMENTO -----------------  


\definecolor{azul}{rgb}{0.0039, 0.3098, 0.6196}


% Formato para el indice general ...........
\makeatletter
    \renewcommand{\@dotsep}{1.5}
    \renewcommand{\l@section}{\@dottedtocline{1}{1.5em}{2.3em}}
    \renewcommand{\l@subsection}{\@dottedtocline{2}{3.8em}{3.2em}}
    \renewcommand{\l@subsubsection}{\@dottedtocline{3}{7.0em}{4.1em}}
\makeatother

% --------- COMANDOS PERSONALIZADOS PARA LA PORTADA DE LAS TAREAS, TRABAJOS Y PROYECTOS ---------

\newcommand{\rutaLogo}[1]{\newcommand{\RutaLogo}{#1}}
\newcommand{\tema}[1]{\newcommand{\Tema}{#1}}
\newcommand{\etiquetaAutores}[1]{\newcommand{\EtiquetaAutores}{#1}}
\newcommand{\alumno}[1]{\newcommand{\Alumno}{#1}}
\newcommand{\materia}[1]{\newcommand{\Materia}{#1}}
\newcommand{\docente}[1]{\newcommand{\Docente}{#1}}
\newcommand{\ciclo}[1]{\newcommand{\Ciclo}{#1}}
\newcommand{\fecha}[1]{\newcommand{\Fecha}{#1}}
\newcommand{\periodo}[1]{\newcommand{\Periodo}{#1}}



% -------------------- DEFINICIÓN DE LA PORTADA --------------------
\rutaLogo{../../../RecursosGlobales/Img/logo_tec_azuay.png}
\tema{\\ \vspace{1cm} Actividad N°1: Taller de conjuntos - Problemas \\ \vspace{1.7cm}}
\etiquetaAutores{Alumno:}
\alumno{Eduardo Mendieta \vspace{1cm}}
\materia{Matemática \vspace{1cm}}
\docente{Lcda. Vilma Duchi \vspace{1cm}}
\ciclo{Primer Ciclo \vspace{1.1cm}}
\fecha{31 de mayo de 2024 \vspace{1cm}}
\periodo{Abril 2024 - Agosto 2024}



% -------------------- INFORME --------------------
\begin{document}

    \begin{titlepage}

    \centering

    \includegraphics[width=0.11\textwidth]{\RutaLogo} 

    \vspace{0.3cm}
    \textcolor{azul}{\Large \textbf{Instituto Superior Universitario Tecnológico del Azuay \\}}
    \vspace{0.3cm}
    \textcolor{azul}{\Large \textbf{Tecnología Superior en Big Data}}
    
    % 1. ---------------- TEMA -------------------------
    
    {\Large\textbf{\Tema}}
    
    % 2. ---------------- AUTOR(ES) -------------------------
    \textcolor{azul}{\large \textbf{\EtiquetaAutores} \\}
    \vspace{0.3cm}
    {\large \Alumno}

    % 3. ---------------- MATERIA -------------------------
    \textcolor{azul}{\large \textbf{Materia:} \\}
    \vspace{0.3cm}
    {\large \Materia}


    % 3. ---------------- DOCENTE -------------------------
    \textcolor{azul}{\large \textbf{Docente:} \\}
    \vspace{0.3cm}
    {\large \Docente}


    % 3. ---------------- Ciclo -------------------------
    \textcolor{azul}{\large \textbf{Ciclo:} \\}
    \vspace{0.3cm}
    {\large \Ciclo}


    % 3. ---------------- FECHA -------------------------
    \textcolor{azul}{\large \textbf{Fecha:} \\}
    \vspace{0.3cm}
    {\large \Fecha}

    % 3. ---------------- PERIODO -------------------------
    \textcolor{azul}{\large \textbf{Periodo Académico:} \\}
    \vspace{0.3cm}
    {\large \Periodo}
 
\end{titlepage}

  
    \section*{\centering Actividad en clase N°1}

        \begin{enumerate}
            % Ejercicio 1: ------------------------------------------------------------------
            \item \textbf{En un club deportivo, el 80\% de los socios juegan al fútbol y el 40\% al baloncesto. Sabiendo que el 30\% de los socios practican los 2 deportes, calcula la probabilidad de que un socio elegido al azar:\\a) Juegue sólo al fútbol.\\b) Juegue sólo al baloncesto.\\c) Juegue al fútbol o al baloncesto.\\d) No juegue a ninguno de los dos deportes.}
                
                \vspace{1cm}
                \begin{venndiagram2sets}[labelA = F, labelB = B, labelNotAB = a, labelOnlyA = b, labelAB = c, labelOnlyB = d, tikzoptions = {scale = 1.5}]
                    
                \end{venndiagram2sets}

                \begin{enumerate}
                    \item $U = 100\%, b + c = 80\%, c + d = 40\%, c = 30\%$
                    \item $b + c = 80\% \rightarrow b + 30\% = 80\% \rightarrow b = 50\%$
                    \item $c + d = 40\% \rightarrow 30\% + d = 40\% \rightarrow d = 10\%$
                    \item $U = 100\% \\a + b + c + d = U \rightarrow a + 50\% + 30\% + 10\% = 100\% \rightarrow a + 90\% = 100\% \rightarrow a = 10\%$
                    \item $F - B = b = \textbf{50\%}$
                    \item $B - F = d = \textbf{10\%}$
                    \item $F \cup B = b + c + d = 50\% + 30\% + 10\% = \textbf{90\%}$
                    \item $U - (F \cup B) = a = \textbf{10\%}$
                \end{enumerate}

                \vspace{1cm}
                \begin{venndiagram2sets}[labelA = F, labelB = B, labelNotAB = 10\%, labelOnlyA = 50\%, labelAB = 30\%, labelOnlyB = 10\%, tikzoptions = {scale = 1.5}]
                    
                \end{venndiagram2sets}

                \textbf{Respuesta:}

                \begin{enumerate}
                    \item El 50\% juegan sólo fútbol.
                    \item El 10\% juegan sólo baloncesto.
                    \item El 90\% juegan al fútbol o al baloncesto.
                    \item El 10\% no juegan ningún deporte.
                \end{enumerate}

                
            % Ejercicio 2: ----------------------------------------------------------------------------------
            \item \textbf{En un grupo de 30 estudiantes pertenecientes a un curso, 15 no estudiaron Matemáticas y 19 no estudiaron Lenguaje. Si tenemos un total de 12 alumnos que no estudiaron Lenguaje ni Matemáticas. ¿Cuántos alumnos estudian exactamente una de las materias mencionadas?}
            
                \vspace{1cm}
                \begin{venndiagram2sets}[labelA = M, labelB = L, labelNotAB = a, labelOnlyA = b, labelAB = c, labelOnlyB = d, tikzoptions = {scale = 1.5}]
                    
                \end{venndiagram2sets}

                \begin{enumerate}
                    \item $U = 30 \rightarrow a + d = 15 \rightarrow a + b = 19 \rightarrow a = 12$
                    \item $a + d = 15 \rightarrow 12 + d = 15 \rightarrow d = 3$
                    \item $a + b = 19 \rightarrow 12 + b = 19 \rightarrow b = 7$
                    \item $M \mathbin{\triangle} L = b + d = 7 + 3 = \textbf{10}$
                \end{enumerate}

                \vspace{1cm}
                \begin{venndiagram2sets}[labelA = M, labelB = L, labelNotAB = 12, labelOnlyA = 7, labelAB = 8, labelOnlyB = 3, tikzoptions = {scale = 1.5}]
                    
                \end{venndiagram2sets}


                \textbf{Respuesta:} 10 alumnos estudian exactamente una de las materias mencionadas.

               
            % Ejercicio 3: ----------------------------------------------------------------------------
            \item \textbf{En una investigación hecha a un grupo de 100 estudiantes, la cantidad de personas que estudian idiomas fueron las siguientes: español, 28; alemán, 30; y francés, 42; español y alemán, 8; español y francés, 10; alemán y francés, 5; los 3 idiomas, 3.\\a) ¿Cuántos no estudian nungún idioma?\\b) ¿Cuántos estudiantes tenían el francés como único idioma de estudio?}
            
                \vspace{1cm}
                \begin{venndiagram3sets}[labelNotABC = a, labelA = E, labelB = A, labelC = F, labelABC = f, labelOnlyAB = c, labelOnlyAC = e, labelOnlyBC = g, labelOnlyA = b, labelOnlyB = d, labelOnlyC = h, tikzoptions = {scale = 1.5}]
                    
                \end{venndiagram3sets}

                \begin{enumerate}
                    \item $U = 100, b + c + e + f = 28, c + d + f + g = 30, e + f + g + h =  42$
                    \item $c + f = 8, e + f = 10, f + g = 5, f = 3$
                    \item $f + g = 5 \rightarrow 3 + g = 5 \rightarrow g = 2$
                    \item $e + f = 10 \rightarrow e + 3 = 10 \rightarrow e = 7$
                    \item $c + f = 8 \rightarrow c + 3 = 8 \rightarrow c = 5$
                    \item $b + c + e + f = 28 \rightarrow b + 5 + 7 + 3 = 28 \rightarrow b + 15 = 28 \rightarrow b = 13$
                    \item $c + d + f + g = 30 \rightarrow 5 + d + 3 + 2 = 30 \rightarrow d + 10 = 30 \rightarrow d = 20$
                    \item $e + f + g + h = 42 \rightarrow 7 + 3 + 2 + h = 42 \rightarrow 12 + h = 42 \rightarrow h = 30$
                    \item $U = a + b + c + d + e + f + g + h = 100\\a + 13 + 5 + 20 + 7 + 3 + 2 + 30 = 100 \rightarrow a + 80 = 100 \rightarrow a = 20$
                    \item $U - (E \cup A \cup F) = a = \textbf{20}$
                    \item $(F - E) - A = h = \textbf{30}$
                \end{enumerate}

                \vspace{1cm}
                \begin{venndiagram3sets}[labelNotABC = 20, labelA = E, labelB = A, labelC = F, labelABC = 3, labelOnlyAB = 5, labelOnlyAC = 7, labelOnlyBC = 2, labelOnlyA = 13, labelOnlyB = 20, labelOnlyC = 30, tikzoptions = {scale = 1.5}]
                    
                \end{venndiagram3sets}

                \textbf{Respuesta:}

                \begin{enumerate}
                    \item 20 alumnos no estudian ningún idioma.
                    \item 30 estudiantes tenían el francés como único idioma de estudio.
                \end{enumerate} 

            \newpage
            % Ejercicio 4: -------------------------------------------------------------------------------------
            \item \textbf{En una reunión se determina que 40 personas son aficionadas al juego, 39 son aficionadas al vino y 48 a las fiestas, además hay 10 personas que son aficionadas al vino, juego y fiestas, existen 9 personas aficionadas al juego y vino solamente, hay 11 personas que son aficionadas al juego solamente y por último 9 a las fiestas y al vino solamente.\\Determinar:\\a) El número de personas que es aficionada al vino solamente.\\b) El número de personas que es aficionada a las fiestas solamente.}
            
                \vspace{1cm}
                \begin{venndiagram3sets}[labelNotABC = a, labelA = J, labelB = V, labelC = F, labelABC = f, labelOnlyAB = c, labelOnlyAC = e, labelOnlyBC = g, labelOnlyA = b, labelOnlyB = d, labelOnlyC = h, tikzoptions = {scale = 1.5}]
                    
                \end{venndiagram3sets}
                

                \begin{enumerate}
                    \item $b + c + e + f = 40, c + d + f + g = 39, e + f + g + h = 48$
                    \item $f = 10, c = 9, b = 11, g = 9$
                    \item $b + c + e + f = 40 \rightarrow 11 + 9 + e + 10 = 40 \rightarrow e + 30 = 40 \rightarrow e = 10$
                    \item $c + d + f + g = 39 \rightarrow 9 + d + 10 + 9 = 39 \rightarrow d + 28 = 39 \rightarrow d = 11$
                    \item $e + f + g + h = 48 \rightarrow 10 + 10 + 9 + h = 48 \rightarrow 29 + h = 48 \rightarrow h = 19$
                    \item $(V - J) - F = d = \textbf{11}$
                    \item $(F - J) - V = h = \textbf{19}$
                \end{enumerate}

                \vspace{1cm}
                \begin{venndiagram3sets}[labelNotABC = 0, labelA = J, labelB = V, labelC = F, labelABC = 10, labelOnlyAB = 9, labelOnlyBC = 9, labelOnlyA = 11, labelOnlyB = 11, labelOnlyAC = 10, labelOnlyC = 19, tikzoptions = {scale = 1.5}]
                
                \end{venndiagram3sets}

                \textbf{Respuesta:}

                \begin{enumerate}
                    \item 11 personas son aficionadas al vino solamente.
                    \item 19 personas son aficionadas a las fiestas solamente.
                \end{enumerate}

        \end{enumerate}

\end{document}