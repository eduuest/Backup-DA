\documentclass[12pt]{article}


% -------------------- PAQUETES --------------------
\usepackage[utf8]{inputenc}
\usepackage[spanish]{babel}
\usepackage[margin=2.54cm]{geometry}
\usepackage{graphicx}
\usepackage{xcolor}
\usepackage{enumitem}


% -------------------- CARGA DE ARCHIVOS EXTERNOS --------------------
% ----------------- UTILIDADES PARA DAR UN MEJOR FORMATO DE DOCUMENTO -----------------  


\definecolor{azul}{rgb}{0.0039, 0.3098, 0.6196}


% Formato para el indice general ...........
\makeatletter
    \renewcommand{\@dotsep}{1.5}
    \renewcommand{\l@section}{\@dottedtocline{1}{1.5em}{2.3em}}
    \renewcommand{\l@subsection}{\@dottedtocline{2}{3.8em}{3.2em}}
    \renewcommand{\l@subsubsection}{\@dottedtocline{3}{7.0em}{4.1em}}
\makeatother

% --------- COMANDOS PERSONALIZADOS PARA LA PORTADA DE LAS TAREAS, TRABAJOS Y PROYECTOS ---------

\newcommand{\rutaLogo}[1]{\newcommand{\RutaLogo}{#1}}
\newcommand{\tema}[1]{\newcommand{\Tema}{#1}}
\newcommand{\etiquetaAutores}[1]{\newcommand{\EtiquetaAutores}{#1}}
\newcommand{\alumno}[1]{\newcommand{\Alumno}{#1}}
\newcommand{\materia}[1]{\newcommand{\Materia}{#1}}
\newcommand{\docente}[1]{\newcommand{\Docente}{#1}}
\newcommand{\ciclo}[1]{\newcommand{\Ciclo}{#1}}
\newcommand{\fecha}[1]{\newcommand{\Fecha}{#1}}
\newcommand{\periodo}[1]{\newcommand{\Periodo}{#1}}



% -------------------- DEFINICIÓN DE LA PORTADA --------------------
\rutaLogo{../../../RecursosGlobales/Img/logo_tec_azuay.png}
\tema{\\ \vspace{1cm} Taller de ejercicios - Algebra booleana  \\ \vspace{1.7cm}}
\etiquetaAutores{Alumno:}
\alumno{Eduardo Mendieta \vspace{1cm}}
\materia{Matemática \vspace{1cm}}
\docente{Lcda. Vilma Duchi, Mgtr. \vspace{1cm}}
\ciclo{Primer Ciclo - M1A \vspace{1.1cm}}
\fecha{27 de junio de 2024 \vspace{1cm}}
\periodo{Abril 2024 - Agosto 2024}



% -------------------- INFORME --------------------
\begin{document}

    \begin{titlepage}

    \centering

    \includegraphics[width=0.11\textwidth]{\RutaLogo} 

    \vspace{0.3cm}
    \textcolor{azul}{\Large \textbf{Instituto Superior Universitario Tecnológico del Azuay \\}}
    \vspace{0.3cm}
    \textcolor{azul}{\Large \textbf{Tecnología Superior en Big Data}}
    
    % 1. ---------------- TEMA -------------------------
    
    {\Large\textbf{\Tema}}
    
    % 2. ---------------- AUTOR(ES) -------------------------
    \textcolor{azul}{\large \textbf{\EtiquetaAutores} \\}
    \vspace{0.3cm}
    {\large \Alumno}

    % 3. ---------------- MATERIA -------------------------
    \textcolor{azul}{\large \textbf{Materia:} \\}
    \vspace{0.3cm}
    {\large \Materia}


    % 3. ---------------- DOCENTE -------------------------
    \textcolor{azul}{\large \textbf{Docente:} \\}
    \vspace{0.3cm}
    {\large \Docente}


    % 3. ---------------- Ciclo -------------------------
    \textcolor{azul}{\large \textbf{Ciclo:} \\}
    \vspace{0.3cm}
    {\large \Ciclo}


    % 3. ---------------- FECHA -------------------------
    \textcolor{azul}{\large \textbf{Fecha:} \\}
    \vspace{0.3cm}
    {\large \Fecha}

    % 3. ---------------- PERIODO -------------------------
    \textcolor{azul}{\large \textbf{Periodo Académico:} \\}
    \vspace{0.3cm}
    {\large \Periodo}
 
\end{titlepage}

  
    \section*{\centering Algebra booleana - Funciones y simplificación de expresiones}

        \vspace{0.6cm}
        
        \begin{enumerate}[label = \textbf{\alph*.}]
            % PARTE 1 -----------------------------------------------------------------
            \item \textbf{Simplifique las siguientes expresiones:}
                \begin{enumerate}[label = \textbf{\arabic*.}]
                    % Ejercicio 1: ***********************************
                    \item $\mathbf{\overline{(A + \overline{B} \cdot C)} + (\overline{A} \cdot B \cdot \overline{C})}$
                    
                    % Ejercicio 2: ***********************************
                    \item $\mathbf{(A \cdot B \cdot \overline{C}) + (A \cdot B \cdot C) + (A \cdot \overline{B} \cdot \overline{C}) + (\overline{A}\cdot B \cdot \overline{C})}$
                    
                    % Ejercicio 3: ***********************************
                    \item $\mathbf{\overline{\overline{A} \cdot (C + D) + \overline{B} \cdot (A + D) + (\overline{A} \cdot \overline{B} \cdot \overline{C})}}$
                    
                    % Ejercicio 4: ***********************************
                    \item $\mathbf{A \cdot (\overline{C} + B \cdot \overline{D} + D \cdot E) + D \cdot (B \cdot C + \overline{A} + B) + \overline{B} \cdot [A \cdot (E + C \cdot E) + (A \cdot \overline{C} \cdot \overline{D} \cdot E)]}$
                    
                \end{enumerate}

            % PARTE 2 ------------------------------------------------------------------
            \item \textbf{Realice las tablas de verdad para las siguientes funciones de salida:}
                \begin{enumerate}[label = \textbf{\arabic*.}]
                    % Ejercicio 1: ***********************************
                    \item $\mathbf{F = A \cdot B + A \cdot \overline{B}}$
                    
                    % Ejercicio 2: ***********************************
                    \item $\mathbf{F = A \cdot B + C \cdot \overline{B}}$
                    
                    % Ejercicio 3: ***********************************
                    \item $\mathbf{F = \overline{\overline{(A + B)} + c}}$
                    
                    % Ejercicio 4: ***********************************
                    \item $\mathbf{Z = \overline{\overline{(A + B)} + \overline{(\overline{B} + C)} + \overline{(B + \overline{C})}}}$
                    
                    % Ejercicio 5: ***********************************
                    \item $\mathbf{Z = (A \cdot \overline{B} \cdot \overline{C}) + (\overline{A} \cdot \overline{B} \cdot C) + (A \cdot \overline{B} \cdot C) + (\overline{A} \cdot B \cdot C)}$
                    
                    % Ejercicio 6: ***********************************
                    \item $\mathbf{F = (A \cdot B \cdot C) + (\overline{A} \cdot B \cdot C) + (B \cdot C)}$
                    
                    % Ejercicio 7: ***********************************
                    \item $\mathbf{Z = (\overline{A} \cdot B \cdot C) + (A \cdot \overline{B} \cdot C) + (A \cdot B \cdot \overline{C}) + (A \cdot B \cdot C)}$
                \end{enumerate}

        \end{enumerate}

\end{document}